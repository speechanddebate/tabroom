\documentclass[12pt]{report} 
\usepackage {fullpage} 
\usepackage{setspace}
\usepackage{lmodern}
\usepackage{fancyhdr,lastpage}
\usepackage[T1]{fontenc}
\usepackage[urw-garamond]{mathdesign}
\pagestyle{fancy}
\fancyhf{} % clear all header and footer fields
\fancyfoot[R]{\footnotesize Page \thepage\ of \pageref{LastPage}}
\fancyfoot[L]{\footnotesize iDebate Tabroom, a free online tournament management system}
\renewcommand{\headrulewidth}{0pt}
\renewcommand{\footrulewidth}{0pt}
\author{Chris Palmer}
\begin{document} 
\normalsize

\title{Managing Tournaments with iDebate Tabroom} \maketitle

\date

\tableofcontents

\setlength{\parskip}{\baselineskip}%
\setlength{\parindent}{10pt}%
\newpage
\onehalfspacing

\chapter{Welcome to iDebate Tabroom}

Welcome to the Tabroom system.  

The purpose of Tabroom is to provide a free method of bringing tournaments
online in every aspect, supporting as many formats of debate and public
speaking as is possible.

The system will conduct pre-tournament registration for you, allowing schools
or institutions to enter their debaters or speakers for your tournament ahead
of time.  Tabroom will manage judging requirements, hired judging requests,
entry limits \& waitlists, housing requests, tournament fees, and deadlines.
It can host your tournament website and allow you to send email updates to part
or all of your registering schools.

During the tournament itself, Tabroom can fully schedule and tabulate an
individual events style tournament or a British-style parliamentary debate
contest (4 team debates, WUDC style).  These features run entirely online and
require only a browser and an internet connection to run.

Tabroom can also provide online support features for other forms of debate,
notably US style policy, LD and Public Forum debate, including pairings \&
results posting online as well as online ballot entry.  Tabroom does not yet
support direct scheduling or tabulation of results in non-WUDC debate events;
however the CAT 2.0 desktop computer program integrates seamlessly with Tabroom
to provide that functionality.  The CAT is available for download directly from
the Tabroom site.  Tabroom can also export registration to, and import
tournament pairings from, Rich Edwards' TRPC software.

Tabroom will pair and post Congress schematics as well but it does not support
tabulation; mostly because Congress rules vary so wildly from tournament to
tournament so they're hard to keep pace with.

The Tabroom system is free \& open source software.  Anyone may download,
modify or use the software under the terms of the GNU General Public License
v2.  Furthermore, the online site at http://www.tabroom.com is free of charge
for anyone to use.  Source code is available at http://svn.tabroom.com.

iDebate Tabroom began as two projects; the CAT (Computer Assisted Tab) and
debateresults.com by Jon Bruschke and tabroom.com by Chris Palmer.

Development of the integrated tabulation system and features are supported by a
grant from the Open Society Foundations.  iDebate Tabroom is now hosted by the
International Debate Education Association (IDEA), and is growing more
integrated with IDEA's other online offerings.  Learn more about IDEA at
http://www.idebate.org.

\medskip

The format of this manual is similar to the format of the software.  When you
log into the tournament director's side of things, you will have a row of drop
down menus across the top of the window.    As your tournament progresses, you
will find yourself marching from left to right and top to bottom in each of
these menus; this is not by accident.   If you're ever unsure of where to go
next, the best rule of thumb is to use the next option over from the one you
were just using; it may just have what you were looking for.

\chapter{The Front Page} 

	When you first go to Tabroom.com, you'll be presented with a homepage that
	mostly has various ways of finding tournaments on it.  On the left will be
	a list of upcoming tournaments, organized by dates; the lighter ones are
	open for online regsitration; while the darker ones are closed (or not yet
	open).  On the right are various ways to search for tournaments; you can
	search by name in the Search Tournaments box above, or you can search by
	location or dates over on the right.

	\section{Circuits}

	Tabroom tournaments are members of one or more 'circuits' to aid in finding
	them.  A circuit will be league, or a regional grouping of tournaments that
	all share roughly the same constituency of institutions attending them.
	They're mostly useful for finding and presenting tournametns for
	registration; the names of the circuits for each tournament are listed on
	the left of the tournament names on the main Tabroom homepage.  You can
	click on a circuit name to see the calendar of events for only that
	circuit.

	\section{Account \& Logging In} 

	To access the full features of Tabroom, you'll need an account, and to log
	in with that account.  Click on Login in the upper right hand corner and
	log in with your email and password.  If you have forgotten your password,
	click on the link below the login information called "Forgot your password?"
	and the system will email you a method of resetting your password.  

	If you haven't created a tabroom account, use the Sign Up link on the top
	right hand side besides Login to create one.  When you create an account or
	when you sign in, Tabroom will bring you to your Homepage.

\chapter{The Homepage} 

	The homepage is your nerve center for everything your accuont can do on
	Tabroom.  You can always reach your homepage by clicking Homepage on the
	top right of any page within Tabroom; or by clicking on your name in the
	top right corner.  The homepage will have a column on the right with all of
	your various connections to entries and tournaments in Tabroom listed out.

	\section{Institutions}

	Under Institutions, you'll find any school, teams, universities, etc that
	you are responsible for registering and managing.  You can create a new one
	under Your Account, or you can join an existing one by being added by
	another school administrator.  You may also have limited access to an
	institution to only enter judge preference sheets; in which case "Prefs"
	will appear next to the school name here on the link with the institution's
	name.

	If you are the administrator for an institution, that institution's details
	will be the default screen you see when you log into Tabroom.  You will see
	a series of tabs which contain information about your institution's attendance
	at tournaments, starting with tournaments.

	\subsection{Tournaments}

		The tournament listing will first show any upcoming tournaments which
		the selected institution is registered for;  you can click Entry to see
		and change the details of your registration (as long as the
		registration deadlines have not expired). 

		Underneath existing tournament registrations, you will see a listing of
		upcoming tournaments for your institution's circuits.  You can enter
		them by clicking Register on the right; or you can ignore tournaments
		you are not interested in by using the yellow [X] symbol on the left.
		To see a tournament you have ignored, click Show Ignored Tournaments on
		the bottom of the screen.
		
		You will not see listings for tournaments not in your circuit; however,
		you can register for other tournaments while you're logged in by going
		back to the Tabroom home page (or click Home on the top left) and
		finding the tournament on the list; when you are logged in a Register
		tab will appear for each tournament and you can log in that way.

	\subsection{Students}

		The Students tab contains your student roster tools.  You enter the
		students or debaters here for later entry into tournaments.  Students
		are entered by name \& graduation year here.  They can be marked as
		"retired" if they should leave your team or you otherwise do not want
		to register them for tournaments; students whose graduation year has
		passed will be marked as 'retired' automatically as well.

		Click on "Show Graduates" on the right to see all students, including
		retired students, if you want to un-retire them.   If you should end up
		with more than one record for a given competitor, click on
		"De-duplicate" on the right to merge identical records.

		Student records can also be linked to a student's Tabroom account.  A
		student might do so in order to automatically recieve text \& email
		messages for pairings, to have access to his/her competitive record, or
		to be able to view RFDs and comments from online ballots.  Students can
		also enter their own pref sheets if you've enabled it under Settings,
		and will soon be able to sign up for tournaments directly (subject to
		coach approval).

		The Students tab will turn red when tabroom accounts have requested to
		link themselves to your student records; approval dialogs will then
		appear at the top of the screen.

	\subsection{Judges}

		The Judges tab is a parallel to the Students tab but for judges and not
		competitors.  Judge roster placements do not automatically expire and
		do not list by graduation date; but you can "retire" judges to trim
		down the size of your roster as well.

	\subsection{Circuits}

		The Circuits tab shows what circuits your institution is a part of.
		You can join on two bases; Full Membership and Tournaments Only.  The
		latter will enable you to recieve any communications or emails that
		circuit administrators send out; and if your circuit manages dues
		through Tabroom will qualify you for those.

		Tournaments Only are for circuits where you do not want any
		communciations and only want to register for tournaments; a circuit
		where you only attend 1-2 tournaments a year would be good for that.
		Not all circuits are set up to offer both types of access.

	\subsection{Settings}

		Settings are where you can set the name of your institution, together
		with various location data, whether or not your entries can enter their
		own pref sheets, and which Tabroom accounts have administrative access
		to register \& change your institution's data.

	\subsection{Results}

		Results are where you find past records of tournaments of your entries,
		judges, and also where you can print invoices and receipts from past
		tournaments, and old registration rosters, for reference.

	\section{Tournaments}

		Tournaments are where you'll find links to the administrative interface
		of present \& future tournaments that you have admin access to.  You
		can also see past tournaments' administrative interfaces under Show
		All.  

		To create a tournament, see Your Account, below.

	\section{Circuits}

		Circuits include features if you are the adminstraive contact for a
		given circuit.  You can click on the name of the circuit to see methods
		of emailing the circuit's membership, setting up circuit-specific
		school abbreviation codes, and approving a tournament for listing on
		your circuit's calendar.  Tournament approvals will appear in yellow
		when they are pending; otherwise all other functions are to be found by
		clicking on the circuit name directly.

	\section{Judging}

		Under judging you will find features that appear if you are linked to
		one or more judging records for a tournament.  You can see your past
		rounds and decisions, together with any upcoming tournaments you've
		been registered for and any pending ballots at tournaments you are at.

	\section{Entries}

		Under entries, if your account is linked to a student entry, you can
		see upcoming tournaments you're regstered for, together with past
		tournament results and records, as well as the ability to enter your
		pref sheets.

	\section{Your Account}

		Your Account is how you get the above features going if you haven't
		already.

		\subsection{Create a new school/team}
			
			Use this section to create a new team or institution if you haven't
			already, or if you are in charge of a second one.  The system will
			ask you for information about name \& location, and then the
			institution will appear on your homepage.  If an institution with
			the same name \& location already exists, however, Tabroom will not
			create it, and instead prompt you to access the existing record;
			you can do this by either contacting a previous coach, or a circuit
			administrator, or the Tabroom helpline (click Help at the top of
			the screen).

		\subsection{Request a new tournament}

			Use this link to create your own tournament if you are hosting
			or directing a tournament.   More details about this process are
			under "Creating a tournament" below.

		\subsection{Link your account to a judge}

			Use this link if you are judging for a program or a tournament and
			want to access your records or online ballots.  Search for your
			name and tournaments/institutions you are judging for here to
			request a link.  Please link only to your \emph{own} judging
			records; not those of your program or institution.  You can sign up
			to get email or text updates for your institution without linking
			your account to a judge; but if you link yourself you preclude that
			judge's ability to enter ballots online.

		\subsection{Link your account to a student}

			Use this link if you are competing for a program or a tournament
			and want to access your records or online ballots.  Search for your
			name and tournaments/institutions you are judging for here to
			request a link.  Please link only to your \emph{own} entry records;
			not those of your program or institution.  You can sign up to get
			email or text updates for your institution without linking your
			account to a competitor; but if you link yourself you preclude that
			competitor's ability to view their ballots or records.

		\subsection{Get updates for a student} 

			Use this link if you want to set up a standing request to follow a
			student at future tournaments; such as if you are that student's coach.

\chapter{Creating or Importing Tournaments}

	The link for creating a tournament will step you through the initial steps
	needed to host and register a tournament on Tabroom.  On the first screen,
	begin by inputting the name, dates and time zone for your tournament.

	The Web name of a tournament is important; this is the name used to help
	users find your tournament website on Tabroom.  If you choose the web name
	"tinfoilhat" for the Conspiracy Theory Classic Tournament, then your tournament
	information will be visible from the shortcut 

	\begin{verbatim}
	http://tinfoilhat.tabroom.com
	\end{verbatim}
	
	It behooves you to choose a short and pithy name, since
	
	\begin{verbatim}
	http://theconspiracytheoryclassictournament2012.tabroom.com
	\end{verbatim}

	is hardly something people are going to remember, much less type.  Choose a
	name without the year or other date indicator; you can re-use and recycle
	tournament names year after year so a date signifier doesn't matter.

	\section{Importing settings from other tournaments}

		If you are creating a brand new tournament from scratch that has never
		been hosted on Tabroom before, click Next:  Set Deadlines.

		However, if you have hosted a tournament on Tabroom before that has the
		same (or mostly the same) settings as the one you are creating; you can
		select it from the pull down menu for Clone Tournament, and then hit
		Next.  This feature will automatically import your settings,
		events/divisions, judge groups, and schedules from your previous
		tournament.  It will not import any entries, school/institutions or
		judges.  It will set deadlines to be the same relative to your new
		tournament dates; if you previous year's entry deadline was 1 week
		before the start of the tournament, so too will the new one be.

		You can also upload a tournament result set using the Universal Data
		Structure format below.  Give the tournament a name and such, and then
		import the XML file created from an outside tabbing software on this
		page under "Upload a tournament from an iDebate XML File".  This
		process will pull whatever data and results are captured in that XML
		file into a new tournament.

	\section{Tournament Basics} 

		The process will then ask you for a series of deadlines, all of which
		are explained on the screen; please read carefully so you know what
		stops when.  
		
		The following page asks you to list what circuits your tournament
		should appear under.  Please do not indiscriminately list all the
		circuits; only 3 circuits will appear on the front page and circuit
		administrators will likely get angry with you.   There are a few
		catch-all circuits for 'when in doubt' cases; such as the National
		Circuit for US high school invitational tournaments which draw from
		many regions at once.

		Then finally the process will ask you for location based information
		about your tournament.  First, you will be asked for a location; if you
		previously hosted a tournament in your circuit at your location, you
		can import the rooms directly from the previous tournament into your
		new tournament, saving some time.  This process will preserve room
		notes and remarks about capacity etc.

		You can actually assign multiple sites to a single tournament, which
		allows you to manage which rounds and categories go to which site and
		manage a very large tournament; however that must be done after you
		have created the tournament already.

		Then click Create Tournament.  It will be ready to administer and use.
		You will not immediately appear on the schedule for whatever circuit or
		circuits you chose; those require approval by circuit administrators
		first.

\chapter{Settings \& Setup}

	\section{Names, info \& access}
		\subsection{Access levels}

	\section{General Tabulation Rules}

	\section{Judge Groups}
		\subsection{Judge Ratings}
		\subsection{Preferencing Systems}
		\subsection{Limitations, strikes, scratches}

	\section{Events or Divisions}

	\section{Schedule}

	\section{Sites \& Rooms} 

	\section{Housing}

	\section{Money}

	\section{Website}

\chapter{Pre-Tournament Registration}

	\section{The Search Bar}

	\section{Viewing Schools}

	\section{Event \& Judge Rosters}

	\section{Housing}

	\section{Reports}

	\section{Tracking Changes \& Witch Hunts}

	\section{Importing \& Exporting Data; TRPC}

	\section{Sending Emails \& Reminders}

\chapter{Scheduling the Tournament}

	\section{Checking Schools In}

	\section{Pairing Preliminary Rounds}

		\subsection{All at once}

		\subsection{One at a time}

		\subsubsection{Entries}
		\subsubsection{Judges}
		\subsubsection{Rooms}

		\subsection{Manipulating the schematic} 
		\subsubsection{Adding someone late}
		\subsubsection{Dropping \& rebalancing}

		\subsection{Disaster Checks}

	\section{Schematics}
		\subsection{Viewing schematics}
		\subsection{Printing schematics \& reports}
		\subsection{Publishing \& blasting schematics}

\chapter{Entering \& Tabulating Results}

	\section{Entering Data}
	\subsection{Tab Staff Entry}
	\subsection{Using Online Balloting}
	\subsection{Manually enter assignments}

	\section{Tabbing Operations}
	\subsection{Ballot table management}
	\subsection{Checking Status}
	\subsection{Coding ballots for sorting}
	\subsection{Entering External Sweepstakes Points}

	\section{Advancing to Break Rounds}

	\subsection{Creating pre-set break rounds}
	\subsection{Viewing \& creating the break}
	\subsection{Selecting judges \& rooms}
	\subsection{Publishing \& releasing}

\chapter{Final Results} 

	\section{Printing Awards \& Results} 
	\section{Sweepstakes}
	\section{Publishing to the Web} 
	
\end{document}

